\section{Komplexe Funktionen}
\subsection{Exponential-/Logarithmusfunktion}
\[e^{j\zeta}= \cjs(\zeta) = \cos\zeta + j\sin\zeta\]

\[e^z= e^{z_1 + jz_2} = e^{z_1}\cdot e^{jz_2} = e^{z_1} \cdot \cjs(z_2) \qquad z \in \mathbb{C} \]
Realteil $z_1$ von $e^{z_1+jz_2}$ ist der Streckungsfaktor. Der Imaginärteil $z_2$ ist der zweite Term.

\subsection{Trigonometrie ($z \in \mathbb{C})$}
\[\sin(z) = \frac{e^{jz} -e^{-jz}}{2j} \quad \cos(a) = \frac{e^{jz} +e^{-jz}}{2} \quad \tan(z) = \frac{\sin(z)}{\cos(z)}\]

\subsection{Harmonische Schwingungen}
\[
z(t) = A \cdot e^{j(\omega t + \varphi)} = \underbrace{A \cdot e^{j\varphi}}_\text{Komplexe Amplitude} \cdot \underbrace{e^{j\omega t}}_{\text{Zeitfunktion}}
\]
\noindent Die Überlagerung von harmonischen Schwingungen können mittels Superpositions-Prinzip berechnet werden.

\noindent Beispiel:
\begin{align*}
	y_1 &= A\sin[\omega t + \varphi] &\qquad y_2 &= A\sin[\omega t + (\varphi + 120°)] \\
	    &= A e ^{j(\omega t + j\varphi)} &\qquad  &= A e^{j(\omega t + \varphi - 120)}\\
 	    &= A e ^{j\omega t} \cdot \underbrace{e^{j\varphi}}_{K_1} &\qquad  &= A e^{j\omega t} \cdot \underbrace{e^{j(\varphi - 120)}}_{K_2}\\    
  	\\
  y &= y_1 + y_2 &&= A e^{j\omega t} (K_1 + K_2) 
\end{align*}

\subsection{Logarithmus und Potenzen}
$\mathbb{R}: \ln(1) = 0 \quad \mathbb{C}: \Ln(1)= 2k\pi j$
\[\Ln(z) = \ln(\left|z\right|) + j\cdot arg(z) \]

\noindent\textbf{Achtung:} $\Ln(0) = undef$


\subsection{Abbildungen}
Um eine Abbildung $w$ durch eine Funktion $f$ auszudrücken, werden gegebene Komplexe Zahlen $z_i$ parametrisiert $r(z_i), c(z_i)$ und mit $f$ verrechnet.
\[z = r(z_i) + jc(z_i) \rightarrow f(z) = w \]

\subsubsection{Parametrisierung}
\todo{}

\subsection{Ableitung}
Die Ableitung $f'$ in $\mathbb{C}$ entspricht einer Drehstreckung mit Faktor $f'$. Eine differenzierbare Funktion $f$ ist in allen Punkten mit $f'(z) \neq 0$ winkeltreu. Sie bewirkt dort eine Drehstreckung mit \textbf{Drehwinkel} $\arg(f'(z))$ und \textbf{Streckfaktor} $|f'(z)|$.

\subsection{Potenzfunktion $f(z) = z^n$}
In Polar-Koordinaten Darstellung werden die Winkel um das $n$-Fache verdoppelt, bzw. mit $\sqrt[n]{z}$ halbiert.
\includegraphics[width=\columnwidth]{Images/quadrat_funktion}

\subsection{Linearefunktion $f(z) = az + b$}
$b$ bewirkt eine Translation um den Ortsvektor $b$. $a$ bewirkt eine Drehstreckung mit Winkel $\arg(a)$ und Streckung mit $|a|$.
\includegraphics[width=\columnwidth]{Images/lineare_funktion}

\newpage

\subsection{Kehrwertfunktion $f(z) = \frac{1}{z}$}
Bewirkt eine Kreisspiegelung.

\includegraphics[width=\columnwidth]{Images/kehrwert_funktion}
\noindent Folgende möglichen Fälle sind zu unterscheiden:\\
\includegraphics[width=\columnwidth]{Images/kreisspiegelung}


\subsection{Möbiustransformation $f(z) = \frac{az+b}{cz+d}$}
Die Möbius-Transformation ist eine Kombination aus:
\[
f: z \xmapsto[u=cz+d]{\text{Linear}} u \xmapsto[v=\frac{1}{u}]{\text{Kehrwert}} v \xmapsto[w=\frac{bc-ad}{c}v+\frac{a}{c}]{\text{Linear}}
\]

\includegraphics[width=\columnwidth]{Images/möbius_funktion}

\subsection{Joukowski-Funktion $f(z) = z + \frac{1}{z}$}
Bewegt weit weg vom Koordinatenursprung unwesentlich, hingegen nahe Punkte werden zusammen gedrückt.\\
\includegraphics[width=\columnwidth]{Images/joukowski_funktion}

\subsection{Exponentialfunktion-Funktion $f(z) = e^z$}
Die Exp-Funktion ist $2\pi j$-Periodisch. Die Waagrechten Gitternetzlinien bilden Strahlen vom Koordinatenursprung weg, senkrechte Gitternetzlinien hingegen gehen zu konzentrischen Kreise um den Ursprung über.\\
\includegraphics[width=\columnwidth]{Images/exp_funktion}


\subsection{Trigo-Funktion $f(z) = sin(z)$}
Da bereits $e^z$ $2\pi j$-Periodisch ist, sind auch die Trigonometrischen Funktionen periodisch. Die Gitternetzlinien gehen zu Ellipsen und Hyperbeln über.
\includegraphics[width=\columnwidth]{Images/sinus_funktion}