\section{Fourierreihe}
Die Basis-Funktionen bestehen aus unendlich vielen:
\[\omega_f = \frac{2\pi}{T} = 2\pi v\]
\[
t_n = \sin(n\omega_f t + \phi_n) \qquad (n \in \mathbf{N})
\]
\[
= \sin(\phi_n) \cdot \cos(n\omega_f t) + \cos(\phi_n)\cdot \sin(n\omega_f)
\]
\todo{Di 11.5.21}

\subsection{Fourierkoeffizienten}
Eine periodische Funktion $f$ mit Periode $T > 0$, lässt sich durch eine Reihe von Sinus- und Kosinusfunktionen darstellen,
deren Frequenzen ganzzahlige Vielfache der Grundfrequenz $\omega = 2\pi/T$ sind:
\[f(t) = \frac{a_0}{2} + \sum_{n = 1}^{\infty}[a_n \cdot \cos(n\omega_ft) + b_n \cdot \sin(n\omega_ft)]\]

\noindent Konkret können die Koeffizienten der Funktion $f(t)$ folgendermassen berechnet werden ($n \in [1;\infty]$):
\[
a_n = \frac{2}{T}\int_{0}^{T}f(t) \cdot \cos(n\omega t)dt \qquad b_n = \frac{2}{T}\int_{0}^{T}f(t) \cdot \sin(n\omega t)dt
\]

\noindent Der erste Koeffizient $a_0$ der Funktion $f(t)$ ist der Mittelwert im Intervall $(0; T)$. Bei einer Fallunterscheidung von $f(t)$ müssen Integrale von allen Fällen (mit entsprechenden Grenzen von $f_n$) summiert werden.

\todo{Fr 14.5.21}

\subsection{Tip's}
\todo{Di. 25.5.21}
\begin{enumerate}[nosep]
	\item Symmetrie
	\item Linearität
	\item Zeitspiegelung
	\item Zeitverschiebung
\end{enumerate}
